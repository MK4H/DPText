%%% A template for a simple PDF/A file like a stand-alone abstract of the thesis.

\documentclass[12pt]{report}

\usepackage[a4paper, hmargin=1in, vmargin=1in]{geometry}
\usepackage[a-2u]{pdfx}
\usepackage[utf8]{inputenc}
\usepackage[T1]{fontenc}
\usepackage{lmodern}
\usepackage{textcomp}

\begin{document}

%% Do not forget to edit abstract.xmpdata.

Vzájemná korelace je často používaný nástroj v oboru zpracování signálu, s aplikacemi pro rozpoznávání obrazu, částicovou fyziku, elektronovou tomografii a mnoho dalších oblastí. Pro mnohé z těchto aplikací je výkon vzájemné korelace limitujícím faktorem pro celkový výkon systému z důvodů její výpočetní náročnosti. V této práci analyzujeme algoritmus pro výpočet vzájemné korelace vzhledem k možnostem pro jeho optimalizaci a paralelizaci. Následně implementujeme několik optimalizací algoritmu založeného na definici vzájemné korelace, zaměřené především na paralelizaci pomocí grafických karet (GPU). Přestože tento algoritmus poskytuje mnoho možností pro paralelizaci, je pro jejich využití potřeba vyřešit několik problémů, jako je například nízká aritmetická intenzita algoritmu. Problémy se nadále liší podle typu vstupních dat, mezi které patří korelace jednoho páru vstupů, jednoho vstupu s množinou jiných vstupů případně korelace mnoha vstupů s mnoha jinými vstupy. V závěru práce poté porovnáme námi implementované optimalizace algoritmu založeného na definici vzájemné korelace s asymptoticky rychlejším a často používaným algoritmem založeným na Rychlé Fourierově transformaci (FFT). V závislosti na celkové velikosti vstupních dat dosahuje naše implementace stejné rychlosti jako algoritmus založený na FFT při velikostech vstupních matic od 60x60 po 150x150, kde pro menší matice pak dosahuje zrychlení oproti FFT.
\end{document}
